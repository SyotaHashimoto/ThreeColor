% 三角形三色問題の概要(数学)
\section{Coqにおける三角形三色問題}

% 4.1 三角形三色問題の証明の Coq の実装(=>)
\subsection{十分条件}
補題\ref{lem:tri_suf}をCoqに実装するために論理式の形にしたものが次の定理\ref{thm:tri_suf}である.
\begin{thm}[十分条件] \label{thm:tri_suf}
  $\forall x, n \in \N, (\exists k \in \N, n = 3 ^ k \Imp \WCTF(x,n))$.
\end{thm}
\begin{proof}
  補題\ref{lem:tri_suf} (pp.\pageref{lem:tri_suf}) でも述べたように
  次を証明することで,定理\ref{thm:tri_suf}を示す.\\
  $\forall \cpos, \forall k, n, x, y \in \N,$ $n = 3^k \Imp (Fmix(\cpos) \Imp$ $\TF(cpos,x,y)).$ \\
  これを$k$に関する数学的帰納法を用いて証明する.
  $k=0$のときは$n=1$となるので明らかに成立する.
  次に$k$のとき成立すると仮定して$k+1$のときも成立することを示す.\\
  $3^k$マスずつ離れたマスの色を図\ref{fig:suf_steps}のように表すことにする.
  \begin{figure}[h]
    \centering
    % 3^{k'+1} 段の三角形
\begin{tikzpicture}
  {\normalsize{
      % 3^(k+1)段目
      \node (a0) {$c_{2}$};
      % 2*3^k段目
      \node[above left=0.5cm of a0] (b0) {$c_{8}$};
      \node[above right=0.5cm of a0] (b1) {$c_{9}$};
      % 3^k段目
      \node[above left=0.5cm of b0] (c0) {$c_{5}$};
      \node[above right=0.5cm of b0] (c1) {$c_{6}$};
      \node[above right=0.5cm of b1] (c2) {$c_{7}$};
      % 0段目
      \node[above left=0.5cm of c0] (d0) {$c_{0}$};
      \node[above right=0.5cm of c0] (d1) {$c_{3}$};
      \node[above left=0.5cm of c2] (d2) {$c_{4}$};
      \node[above right=0.5cm of c2] (d3) {$c_{1}$};
      % 点線 [3^(k+1)段 〜 2*3^k 段]
      \node[blue] at ($(a0)!.5!(b0)$) {$\ddots$};
      \node[blue] at ($(b0)!.5!(b1)$) {$\cdots$};
      \node[blue] at ($(a0)!.5!(b1)$) {$\iddots$};
      % 点線 [2*3^k 段 〜 3^k 段]
      \node[teal] at ($(b0)!.5!(c0)$) {$\ddots$};
      \node[teal] at ($(c0)!.5!(c1)$) {$\cdots$};
      \node[teal] at ($(b0)!.5!(c1)$) {$\iddots$};
      \node[red] at ($(b1)!.5!(c1)$) {$\ddots$};
      \node[red] at ($(c1)!.5!(c2)$) {$\cdots$};
      \node[red] at ($(b1)!.5!(c2)$) {$\iddots$};
      % 点線 [3^k 段 〜 0 段]
      \node[red] at ($(c0)!.5!(d0)$) {$\ddots$};
      \node[red] at ($(d0)!.5!(d1)$) {$\cdots$};
      \node[red] at ($(c0)!.5!(d1)$) {$\iddots$};
      \node[blue] at ($(c1)!.5!(d1)$) {$\ddots$};
      \node[blue] at ($(d1)!.5!(d2)$) {$\cdots$};
      \node[blue] at ($(c1)!.5!(d2)$) {$\iddots$};
      \node[teal] at ($(c2)!.5!(d2)$) {$\ddots$};
      \node[teal] at ($(d2)!.5!(d3)$) {$\cdots$};
      \node[teal] at ($(c2)!.5!(d3)$) {$\iddots$};
  }}
\end{tikzpicture}

    \caption{彩色三角形($n=3^{k+1}$のとき)}
    \label{fig:suf_steps}
  \end{figure}
  このとき,$3$つの端点のマスの色が$c_0=cpos(x,y^k)$,$c_3=cpos(x,y)$,
  $c_5=cpos(x+3^k,y^k)$である$3^k$段の彩色三角形が調和彩色三角形であることは,
  帰納法の仮定からすぐに示せる.
  同様にして,この彩色三角形以外にも$5$つの彩色三角形は調和彩色三角形である.
  すなわち,次の$6$つが成立する.
  \begin{itemize}
    \item $\TF(cpos,x,y,3^k)$
    \item $\TF(cpos,x+3^k,y,3^k)$
    \item $\TF(cpos,x+2\cdot3^k,y,3^k)$
    \item $\TF(cpos,x,y+3^k,3^k)$
    \item $\TF(cpos,x+3^k,y+3^k,3^k)$
    \item $\TF(cpos,x,y+2\cdot3^k,3^k)$
  \end{itemize}
  これらと補題\ref{lem:mixCut} $(\mixCut)$ を用いて式変形をすると,
  \[
  \cpos(x,y+3^k)=\mix(\cpos(x,y),\cpos(x+3^{k+1},y))
  \]
  が得られる.すなわち,$\TF(cpos,x,y,3^{k+1})$.\\
  したがって,$\TF(cpos,x,y,n)$である.
\end{proof}

% 4.2 三角形三色問題の証明の Coq の実装(<=)
% n が偶数の場合,n が奇数で短い場合,n が奇数で長い場合
% $n$が偶数のとき
% $n$が奇数 かつ $3^{k'} < n \leq 2 \cdot 3^{k}$のとき
% $n$が奇数 かつ $2 \cdot 3^{k'} + 1 \leq n < 3^{k+1}$のとき

\subsection{必要条件} \label{sec:nec}
補題\ref{lem:tri_nec}をCoqに実装するために論理式の形にしたものが次の定理\ref{thm:tri_nec}である.
\begin{thm}[必要条件] \label{thm:tri_nec}
  $\forall x, n \in \N, n > 0 \Imp$ \\
  $(\WCTF(x,n) \Imp \exists k \in , n = 3^k)$ 
\end{thm}
補題\ref{lem:tri_nec} (pp.\pageref{lem:tri_nec}) でも述べたように
次の対偶を証明することで定理\ref{thm:tri_nec}を示す.\\
$\forall n, x \in \N, n > 0 \Imp$
$(\lnot(\exists k \in \N, n = 3 ^ k) \Imp \lnot\WCTF(x,n))$ \\
この対偶を証明するためには,
\begin{itemize}
\item
  $n > 0$
\item
  $\lnot(\exists k \in \N, n = 3 ^ k)$,
\item
  $\WCTF(x,n)$
\end{itemize}
を仮定して矛盾を示せばよい.
今回は$n$に関する場合分けをしてから各場合において矛盾を導く.
\subsubsection{$n$が偶数の場合}
$n$が偶数のときは補題\ref{lem:evenA},\ref{lem:evenB}を証明してから,補題\ref{lem:even}を証明して矛盾を導く.
\begin{lem}[\EvenA] \label{lem:evenA}
  $\forall \cpos, \forall x, n \in \N, n > 0  \Imp Fmix(\cpos) \Imp 
  (\forall i \in \N, (0 \leq i \leq n \Imp \cpos(x+i,0) = \colorYB(x,n,x+i))) \Imp
  (\forall i \in \N, (0 \leq i \leq n-1 \Imp \cpos(x+i,1) = red))$.
\end{lem}
補題\ref{lem:evenA}は最上段のマスの色を関数$\colorYB$で塗ると,最上段より$1$段下の段のマスの色はすべて赤であることを表している.
\begin{proof}
  $0$ $\leq$ $i$ $\leq$ $n-1$を満たす$i$を任意にとると,
  仮定より$\cpos(x+i,0) = \colorYB(x,n,x+i)$,$\cpos(x+i+1,0) = \colorYB(x,n,x+i+1)$.
  また,$\Fmix(\cpos)$より$\cpos(x+i,1) = \mix(\cpos(x+i,0),\cpos(x+i+1,0))$が導ける.
  \begin{itemize}
  \item
    $i$が偶数のとき \\
    $\colorYB$の定義より,$\colorYB(x,n,x+i)=\blu$,$\colorYB(x,n,x+i+1)=\yel$であるから$\cpos(x+i,1)=\red$.
  \item
    $i$が奇数のとき \\
    $\colorYB$の定義より,$\colorYB(x,n,x+i)=\yel$,$\colorYB(x,n,x+i+1)=\blu$であるから$\cpos(x+i,1)=\red$.
  \end{itemize}
  よって,$i$の遇奇にかかわらず$\cpos(x+i,1)=\red$.
\end{proof}
\begin{lem}[\EvenB] \label{lem:evenB}
  $\forall \cpos, \forall x, n \in \N, n > 0 \Imp \Fmix(\cpos) \Imp 
  (\forall i \in \N, (0 \leq i \leq n \Imp \cpos(x+i,0) = \colorYB(x,n,x+i))) \Imp
  (\cpos(x,n)=\red).$
\end{lem}
補題\ref{lem:evenB}は最上段のマスの色を関数$\colorYB$で塗ると,最下段のマスの色は赤になるということを表している.
\begin{proof}
  補題\ref{lem:evenA}より$\forall i \in \N, (0 \leq i \leq n-1 \Imp \cpos(x+i,1) = red)$.
  さらに,補題\ref{lem:AllRed}より$\cpos(x,n)=\red$.
\end{proof}

\begin{lem}[\Even] \label{lem:even}
  $\forall x,n \in \N, (n > 0 \land odd(n) = false) \Imp \lnot\WCTF(x,n)$.

ただし,補題$\ref{lem:even}$の中にある$odd(n)$は次のようにSSReflectで定義されている関数である.

自然数$n$に対して,
\[
odd(n) \eqDef
\begin{cases}
  true & \text{($n$が奇数)} \\
  false & (otherwise)
\end{cases}
\]
\end{lem}
\begin{proof}
  補題\ref{lem:paint}より
  $\exists \cposYB, \Fmix(\cposYB) \land \forall x_1, y_1 \in \N, \cposYB(x_1,y_1) = \lift(\colorYB(x,$ $n),x_1,y_1)$.
  さらに,存在する $\cposYB$ をそのまま $\cposYB$ として名付けると,
  $\forall i \in \N, \colorYB(x,n,x+i) = \cposYB(x+i,0)$ を満たす.
  また,$0 \leq 0 \leq n$,$0 \leq n \leq n$を満たすので
  $\colorYB(x,n,x)=\colorYB(x,n,x+n)=\yel$.
  さらに,仮定より$\TF(\cpos,x,0,n)$であるから$\cposYB(x,n)=\yel$となる.
  一方で,補題\ref{lem:evenB}より$\cpos(x,n)=\red$となるので矛盾する.
\end{proof}

\subsubsection{$n$が奇数 かつ $3^{k} < n \leq 2 \cdot 3^{k}$の場合}
$n$が奇数 かつ$3^{k'} < n \leq 2 \cdot 3^{k}$のときは補題\ref{lem:shortoddA},\ref{lem:shortoddB},\ref{lem:shortoddC}を証明してから,補題\ref{lem:shortodd}を証明して矛盾を導く.
\begin{lem}[\ShortOddA] \label{lem:shortoddA}
  $\forall \cpos, \forall x, n, k \in \N,
  (3^k < n \leq (3^k\cdot2) \land odd(n) = true) \Imp
  n > 0  \Imp Fmix(\cpos) \Imp 
  (\forall x_1, y_1 \in \N, \TF(\cpos,x_1,y_1,$ $3^k)) \Imp
  (\forall i \in \N, (0 \leq i \leq n \Imp \cpos(x+i,0) = \colorYBBY(x,n,x+i))) \Imp
  (\forall i \in \N, (0 \leq i \leq n - 3^k \Imp \cpos(x+i,3^k) = \colorYB(x,n-3^k,x+i)))$.
\end{lem}
補題\ref{lem:shortoddA}は最上段のマスの色を関数$\colorYBBY$で塗ると,最上段より$3^k$下の段のマスは黄,青で交互に塗ってあることを表している.
\begin{proof}
  $0 \leq i \leq n-3^k$を満たす$i$を任意にとると,
  $0 \leq i \leq n$,$0 \leq i+3^k \leq n$であるから仮定より,
  $\cpos(x+i,0) = \colorYBBY(x,n,x+i)$,
  $\cpos(x+i+3^k,0) = \colorYBBY(x,n,x+i+3^k))$.
  また,仮定の$\TF(\cpos,x+i,0,3^k)$より
  $\cpos(x+i,3^k)=\mix(\cpos(x+i,n),\cpos(x+i+3^k,0))$が成立する.
  さらに,$n$は奇数であり$0 \leq i \leq n/2$,$n/2+1 \leq i+3^k \leq n$を満たすので$\colorYBBY$,$\colorYB$の色は$i$の遇奇によって定まる.
  \begin{itemize}
  \item
    $i$が偶数のとき \\
    $\colorYBBY$の定義より$\colorYBBY(x,n,x+i)=\yel$,$\colorYBBY(x,n,x+i+3^k)=\yel$であり,$\colorYB$の定義より$\colorYB(x,n-3^k,x+i)=\yel$.
    よって,$\cpos(x+i,3^k)=\mix(\yel,\yel)=\yel=\colorYB(x,n-3^k,x+i)$.
  \item
    $i$が奇数のとき \\
    $\colorYBBY$の定義より$\colorYB(x,n,x+i)=\blu$,$\colorYB(x,n,x+i+3^k)=\blu$であり,$\colorYB$の定義より$\colorYB(x,n-3^k,x+i)=blu$
    よって,$\cpos(x+i,3^k)=\mix(\blu,\blu)=\blu=\colorYB(x,n-3^k,x+i)$.
  \end{itemize}
  以上より,$i$の遇奇にかかわらず$\cpos(x+i,3^k) = \colorYB(x,n-3^k,x+i))$.
\end{proof}

\begin{lem}[\ShortOddB] \label{lem:shortoddB}
  $\forall \cpos, \forall x, n, k \in \N,
  (3^k < n \leq 3^k \cdot 2 \land odd(n) = true) \Imp
  n > 0  \Imp Fmix(\cpos) \Imp 
  (\forall x_1, y_1 \in \N, \TF(\cpos,x_1,y_1,$ $3^k)) \Imp
  (\forall i \in \N, (0 \leq i \leq n \Imp \cpos(x+i,0) = \colorYBBY(x,n,x+i))) \Imp
  (\forall i \in \N, (0 \leq i \leq n - 3^k-1 \Imp cpos(x+i,3^k+1)= \red))$.
\end{lem}
補題\ref{lem:shortoddB}は最上段のマスの色を関数$\colorYBBY$で塗ると,最上段から$3^k+1$下の段のマスはすべて赤であるということを表している.
\begin{proof}
  補題\ref{lem:shortoddA}より$\forall i \in \N, (0 \leq i \leq n - 3^k \Imp \cpos(x+i,3^k) = \colorYB(x,n-3^k,x+i))$となるので,
  $\cpos(x+i,3^k) = \colorYB(x,n-3^k,x+i)$,
  $\cpos(x+i+1,3^k) = \colorYB(x,n-3^k,x+i+1)$.
  $\Fmix(\cpos)$より$\cpos(x+i,3^k+1) = \mix(\cpos(x+i,3^k),\cpos(x+i+1,3^k))$.
  ここで,補題\ref{lem:shortoddA}と同様にして $i$ の偶奇で場合分けをする.
  \begin{itemize}
  \item
    $i$が偶数のとき \\
    $\colorYB$の定義より$\colorYB(x,n-3^k,x+i)=\yel$,$\colorYB(x,n-3^k,x+i+1)=\blu$.
    よって,$\cpos(x+i,3^k+1)=\mix(\cpos(x+i,3^k),\cpos(x+i+1,3^k))=\mix(\yel,\blu)=\red$.
  \item
    $i$が奇数のとき \\
    $\colorYB$の定義より$\colorYB(x,n-3^k,x+i)=\blu$,$\colorYB(x,n-3^k,x+i+1)=\yel$.
    よって,$\cpos(x+i,3^k+1)=\mix(\cpos(x+i,3^k),\cpos(x+i+1,3^k))=\mix(\blu,\yel)=\red$.
  \end{itemize}
  以上より,$i$の遇奇にかかわらず$\cpos(x+i,3^k+1) = \red$.
\end{proof}

\begin{lem}[\ShortOddC] \label{lem:shortoddC}
  $\forall \cpos, \forall x, n, k \in \N,
  (3^k < n \leq 3^k \cdot 2 \land odd(n) = true) \Imp
  n > 0  \Imp Fmix(\cpos) \Imp 
  (\forall x_1, y_1 \in \N, \TF(\cpos,x_1,y_1,$ $3^k)) \Imp
  (\forall i \in \N, (0 \leq i \leq n \Imp \cpos(x+i,0) = \colorYBBY(x,n,x+i))) \Imp
  (\forall i \in \N, (0 \leq i \leq n - 3^k-1 \Imp \cpos(x,n)= \red))$.
\end{lem}
補題\ref{lem:shortoddC}は最上段のマスの色を関数$\colorYBBY$で塗ると最下段のマスの色は赤になることを表している.
\begin{proof}
  補題\ref{lem:shortoddB}より$\forall i \in \N, (0 \leq i \leq n - 3^k-1 \Imp cpos(x+i,3^k+1)= \red)$.
  さらに,補題\ref{lem:AllRed}より$\cpos(x,n)=\red$.
\end{proof}

\begin{lem}[\ShortOdd] \label{lem:shortodd}
  $\forall x, n, k \in \N,
  (3^k < n \leq 3^k \cdot 2 \land odd(n) = true) \Imp \lnot\WCTF(x,n).$
\end{lem}
\begin{proof}
  補題\ref{lem:paint}より
  $\exists \cposYBBY, \Fmix(\cposYBBY)$ $ \land \forall x_1, y_1 \in \N, \cposYBBY(x_1,y_1) = \lift(\colorYBBY$ $(x,n),x_1,y_1)$.
  さらに,存在する $\cposYBBY$ をそのまま $\cposYBBY$ として名付けると,
  $\forall i \in \N, \colorYBBY(x,n,x+i) = \cposYBBY(x+i,0)$ を満たす.
  これより$\colorYBBY(x,n,x) = \cposYBBY(x,0)$,$\colorYBBY(x,n,x+n) = \cposYBBY$ $(x+n,0)$.
  $n$が奇数であるから$\colorYBBY(x,n,x)=\colorYBBY(x,n,x+n)$が成立するので,
  $\colorYBBY$ $(x,n,x)=\colorYBBY(x,n,x+n)=\yel$.
  さらに,仮定より$\TF(\cposYBBY,x,0,n)$であるから$\cposYBBY(x,n)=\yel$となる.
  一方で,定理\ref{thm:tri_suf},補題\ref{lem:shortoddC}より$\cpos(x,n)=\red$となるので矛盾する.
\end{proof}


\subsubsection{$n$が奇数 かつ $2 \cdot 3^{k} + 1 \leq n < 3^{k+1}$の場合}
$n$が奇数 かつ$2 \cdot 3^{k} + 1 \leq n < 3^{k+1}$のときは補題\ref{lem:longoddA},\ref{lem:longoddB},\ref{lem:longoddC}を証明してから,補題\ref{lem:longodd}を証明して矛盾を導く.
\begin{lem}[\LongOddA] \label{lem:longoddA}
  $\forall \cpos, \forall x, n, k \in \N,
  (3^k \cdot 2 + 1 \leq n < 3^{k+1}) \Imp
  Fmix(\cpos) \Imp 
  (\forall x_1, y_1 \in \N, \TF(\cpos,x_1,y_1,3^k)) \Imp
  (\forall i \in \N, (0 \leq i \leq n \Imp \cpos(x+i,0) = \colorBYB(x,n,x+i))) \Imp
  (
   (\forall i \in \N,(0 \leq i \leq n - 3^k \cdot 2 \Imp \cpos(x+i,3^k) = \red))
   \land
   (\forall i \in \N,(3^k \leq i \leq n - 3^k \Imp \cpos(x+i,3^k)=\red))
  )$.
\end{lem}
補題\ref{lem:longoddA}は最上段のマスの色を関数$\colorBYB$で塗ると,最上段より$3^k$下の段のマスは外側から$n-2\cdot3^k+1$マスはすべて赤で塗られていることを表している.
\begin{proof}
  $3^k\cdot2 + 1 \leq n < 3^{k+1}$を満たす$n$をとる.
  \begin{itemize}
  \item
    $\forall i \in \N,(0 \leq i \leq n - 3^k \cdot 2 \Imp \cpos(x+i,3^k) = \red)$を示す.\\
    $0 \leq i \leq n - 3^k \cdot 2$を満たすように任意に$i$をとると,
    $0 \leq i \leq n$,$0 \leq i \leq 3^k-1$を満たすので,
    仮定より$\cpos(x+i,0)=\colorBYB(x,n,k,x+i)$であり,$\colorBYB(x,n,k,x+i)=\blu$が導ける.
    よって,$\cpos(x+i,0)=\blu$.
    また,$0 \leq i+3^k \leq n$,$3^k \leq i+3^k \leq n-3^k$を満たすので,
    $\cpos(x+i+3^k,0)=\colorBYB(x,n,k,x+i+3^k)$であり,$\colorBYB(x,n,k,x+i+3^k)=\yel$が導ける.
    よって,$\cpos(x+i+3^k,0)=\yel$.
    さらに,仮定より$\TF(\cpos,x+i,0,3^k)$だから$cpos(x+i,3^k)=\mix(\cpos(x+i,0)=\colorBYB(x,n,k,x+i),\cpos(x+i+3^k,0))=\mix(\blu,\yel)=\red$が成立する.
    よって,$\cpos(x+i,3^k)=\red$.
  \item
    $\forall i \in \N,(3^k \leq i \leq n - 3^k \Imp \cpos(x+i,3^k)=\red))$を示す.\\
    $3^k$ $\leq$ $i$ $\leq$ $n - 3^k$を満たすように任意に$i$をとると,
    $0 \leq i \leq n$,$3^k \leq i \leq n-3^k$を満たすので,
    仮定より$\cpos(x+i,0)=\colorBYB(x,n,k,x+i)$であり,$\colorBYB(x,n,k,x+i)=\yel$が導ける.
    よって,$\cpos(x+i,0)=\yel$.
    また,$0 \leq i+3^k \leq n$,$3^k \leq i+3^k \leq n-3^k$を満たすので,
    $\cpos(x+i+3^k,0)=\colorBYB(x,n,k,x+i+3^k)$であり,$\colorBYB(x,n,k,x+i+3^k)=\blu$が導ける.
    よって,$\cpos(x+i+3^k,0)=\blu$.
    さらに,仮定より$\TF(\cpos,x+i,0,3^k)$だから$cpos(x+i,3^k)=\mix(\cpos(x+i,0),\cpos(x+i+3^k,0))=\mix(\yel,\blu)=\red$が成立する.
    よって,$\cpos(x+i,3^k)=\red$.
  \end{itemize}
\end{proof}

\begin{lem}[\LongOddB] \label{lem:longoddB}
  $\forall \cpos, \forall x, n, k \in \N,
  (3^k \cdot 2 + 1 \leq n < 3^{k+1}) \Imp
  Fmix(\cpos) \Imp 
  (\forall x_1, y_1 \in \N, \TF(\cpos,x_1,y_1,3^k)) \Imp
  (\forall i \in \N, (0 \leq i \leq n \Imp \cpos(x+i,0) = \colorBYB(x,n,x+i))) \Imp
  \forall i \in \N, (0 \leq i \leq n - 3^k \cdot 2 \Imp \cpos(x+i,3^k \cdot 2) = red)$.
\end{lem}
補題\ref{lem:longoddB}は最上段のマスの色を関数$\colorBYB$で塗ると,最上段から$3^k\cdot2$下の段のマスはすべて赤で塗られていることを表している.
\begin{proof}
  $0 \leq i \leq n - 3^k \cdot 2$を満たす$i$を任意にとると,
  補題\ref{lem:longoddA}より$\cpos(x+i,3^k) = red$.
  また,$3^k$ $\leq i+3^k \leq n - 3^k$でもあるから
  補題\ref{lem:longoddA}より$\cpos(x+i+3^k,3^k) =\red$.
  さらに,仮定より$\TF(\cpos,x+i,3^k,3^k)$だから
  $cpos(x+i,3^k \cdot 2)=\mix(\cpos(x+i,3^k),\cpos(x+i+3^k,3^k))=\mix(\red,\red)=\red$.
  よって,$\cpos(x+i,3^k \cdot 2)=\red$.
\end{proof}

\begin{lem}[\LongOddC] \label{lem:longoddC}
  $\forall \cpos, \forall x, n, k \in \N,
  (3^k \cdot 2 + 1 \leq n < 3^{k+1}) \Imp
  Fmix(\cpos) \Imp 
  (\forall x_1, y_1 \in \N, \TF(\cpos,x_1,y_1,3^k)) \Imp
  (\forall i \in \N, (0 \leq i \leq n \Imp \cpos(x+i,0) = \colorBYB(x,n,x+i))) \Imp
  (\cpos(x,n) = \red)$.
\end{lem}
補題\ref{lem:longoddC}は最上段のマスの色を関数$\colorBYB$で塗ると,最下段のマスは赤になることを表している.
\begin{proof}
  補題\ref{lem:longoddB}より$\forall i \in \N, (0 \leq i \leq n - 3^k \cdot 2 \Imp \cpos(x+i,3^k \cdot 2) = red).$
  さらに,補題\ref{lem:AllRed}より$\cpos(x,n)=\red$.
\end{proof}
\begin{lem}[\LongOdd] \label{lem:longodd}
  $\forall x, n, k \in \N,
  (3^k\cdot2 + 1 \leq n < 3^{k+1} \land odd(n) = true) \Imp \lnot\WCTF(x,n).$
\end{lem}
\begin{proof}
  補題\ref{lem:paint}より
  $\exists \cposBYB, \Fmix(\cposBYB) \land \forall x_1, y_1 \in \N, \cposBYB(x_1,y_1) = \lift(\colorBYB$ $(x,n),x_1,y_1)$.
  さらに,存在する $\cposBYB$ をそのまま $\cposBYB$ として名付けると,
  $\forall i \in \N, \cposBYB(x+i,0) = colorBYB(x,n,k,x+i)$ を満たす.
  これより$\cposBYB(x,0) = \colorBYB(x,n,k,x)$,$\cposBYB(x+n,0) = \colorBYB$ $(x,n,k,x+n)$.
  また,$0 \leq 0 \leq 3^k-1$,$n-3^k+1 \leq n \leq n$より
  $\colorBYB(x,n,k,x) = \colorBYB(x,n,k,x+n) = \blu$.
  さらに,仮定より$\TF(\cposBYB,x,0,n)$であるから
  $\cposYBBY(x,n) = \mix(\cposBYB(x,0),\cposBYB(x+n,0)) = \mix(\blu,\blu) = \blu$
  となる.
  一方で,定理\ref{thm:tri_suf},補題\ref{lem:longoddC}より
  $\cpos(x,n)=\red$となるので矛盾する.
\end{proof}

以上より補題\ref{lem:even},\ref{lem:shortodd},\ref{lem:longodd}から
\ref{sec:nec}節の冒頭で述べた定理\ref{thm:tri_nec}を証明された.
さらに,定理\ref{thm:tri_suf}(十分条件),定理\ref{thm:tri_nec}(必要条件)が
成立するので定理\ref{thm:tri_iff}(必要十分条件)も示された.
