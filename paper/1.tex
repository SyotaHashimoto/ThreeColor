% 導入
\section{はじめに}
% Coq + ssreflect の説明
Coq とは数学の定理や補題,主張の正しさを保証するためのソフトウェアの$1$つであり,こういったソフトウェアを定理証明支援系と呼ぶ.
証明の作成中の各場面で示すべき主張(サブゴール)に対して人間がサブゴールを示すための次の一手を指示するとCoqは次のサブゴールを提示し,人間の次の一手を待つ.
このような対話的なやりとりによりCoqは証明の完成の手助けをする.
証明の規模が大きくなると,複雑な場合分けの漏れがあったり計算ミスなど機械的操作のミスにより人間は誤った証明をしてしまうことがある.
 Coq の支援を受けることで,このような誤りが排除された信頼できる証明を得ることができる.
本研究では,Coqの拡張であるSSReflectを採用した.
 SSReflect は数学の定理や補題,主張を記述することのできるコンピュータ上の証明言語である.
また,プログラミング言語という側面ももつため複製が容易にできるという特徴がある.
一度作成された証明自体はCoqのファイルであり,このファイルを公開することで他の人が利用することができる.
このようにして様々な証明を多くの人たちがつかえるようになることで, 公開された証明1つ1つがたとえ小さな証明であっても組み合わせることで証明の難しい定理を示しやすくなる.

三角形三色問題とは$n$段の逆三角形に配置されたすべての正六角形のマスに対して,隣り合う2マスとそれらに接する下の段のマスの色が$3$色とも同じか3色とも異なるように3色で塗り分けたとき,逆三角形の端点の3マスの色も$3$色とも同じか3色とも異なるような段数の一般項を求める問題である.
この問題の解決方法は大きく分けて3つの方法が知られている.\cite{tri}
\begin{enumerate}
\item \label{ans_1}
  パスカルの三角形の値をmod $3$としたものをマスの色と対応させて使い,代数学での $Lucas$ の定理と関係のある命題を示すことで解決する方法.
\item \label{ans_2}
  逆三角形のマスの塗り方が$n$個の独立パターンに分解できること利用して,重ね合わせの原理を用いることで解決する方法.
\item \label{ans_3}
  比較的少ない段数で成立する段数を調べて段数の規則性(数列)を見つけ出し,この数列から予測できる段数の一般項を推測する方法.
\end{enumerate}
\ref{ans_3}の方法において一般項が$3^k$段であると推測されており,推測された一般項の段数でないならば逆三角形の塗り方の規則に従わない反例が存在することも知られている.

% やったこと
本研究では\ref{ans_3}.の方法で推測して得られた一般項が必要十分条件になっていることをCoqを用いて証明した.
Coqに実装するにあたって,三角形三色問題は幾何的な側面を多くもつ問題であるためこのままではCoqにコードとして実装することができない.
そのため,Coqで証明をするためには三角形三色問題を論理式に書き直したり,三角形三色問題をCoqで再現するために適切な公理を用意する必要がある.
もし,適切でない公理を用意してしまうと本来証明できないはずの主張が証明できてしまったり,逆に証明できるはずの主張が証明できてしまう.
また,十分条件の証明の方法としては一般項に現れる自然数$k$に関する数学的帰納法を用いて証明した.
一方で,必要条件は対偶法と$n$について場合分けをすることで証明した.

% もくじ
本論文は
第$2$章では三角形三色問題の証明の概要について述べる.
第$3$章では三角形三色問題の証明をCoqに実装するために必要な準備について述べる.
第$4$章では実際にCoqに実装した三角形三色問題の証明について述べる.
