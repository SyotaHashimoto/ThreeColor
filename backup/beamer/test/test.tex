\documentclass[dvipdfmx,cjk]{beamer} 
%\documentclass[dvipdfm,cjk]{beamer} %% オプションは環境や利用するプログラムに
%\documentclass[dvips,cjk]{beamer}   %% よって変える

\AtBeginDvi{\special{pdf:tounicode 90ms-RKSJ-UCS2}} %% しおりが文字化けしないように
%\AtBeginDvi{\special{pdf:tounicode EUC-UCS2}}

%\setbeamertemplate{navigation symbols}{} %% 右下のアイコンを消す

\usetheme{CambridgeUS}         %% theme の選択
%\usetheme{Boadilla}           %% Beamer のディレクトリの中の
%\usetheme{Madrid}             %% beamerthemeCambridgeUS.sty を指定
%\usetheme{Antibes}            %% 色々と試してみるといいだろう
%\usetheme{Montpellier}        %% サンプルが beamer\doc に色々とある。
%\usetheme{Berkeley}
%\usetheme{Goettingen}
%\usetheme{Singapore}
%\usetheme{Szeged}

%\usecolortheme{rose}          %% colortheme を選ぶと色使いが変わる
%\usecolortheme{albatross}

%\useoutertheme{shadow}                 %% 箱に影をつける
%\usefonttheme{professionalfonts}       %% 数式の文字を通常の LaTeX と同じにする

%\setbeamercovered{transparent}         %% 消えている文字をうっすらと表示する

\setbeamertemplate{theorems}[numbered]  %% 定理に番号をつける
\newtheorem{thm}{Theorem}[section]
\newtheorem{prop}[thm]{Proposition}
\theoremstyle{example}
\newtheorem{exm}[thm]{Example}
\newtheorem{rem}[thm]{Remark}
\newtheorem{que}[thm]{Question}
\newtheorem{prob}[thm]{Problem}

\begin{document}
\title[Beamer]{Beamer の使い方} 
\author[Shinshu]{信州太郎}            %% ここに書かれた情報は色々なところに使われるので
\institute[Shinshu Univ.]{信州大学}   %% なるべく設定した方が良い
\date{February 6, 2007}

\begin{frame}                  %% \begin{frame}..\end{frame} で 1 枚のスライド
\titlepage                     %% タイトルページ
\end{frame}

\begin{frame}                  %% 目次 (必要なければ省略)
\tableofcontents
\end{frame}

\section{箇条書き}             %% セクション名
\begin{frame}
\frametitle{松本}              %% フレームタイトル

\begin{itemize}
\item 豊科\pause               %% \pause でとまる
\item 穂高\pause
\item 明科
\end{itemize}
\end{frame}

\section{定理型環境}           %% 定理型環境が使える
\begin{frame}                  %% \newtheorem で新しい環境も作れる
\begin{thm}
定理型環境が使える。
使い方は普通の \LaTeX と同じ
\end{thm}
\pause

\begin{proof}
証明も書ける。
\end{proof}
\pause

\begin{exm}                   %% 色が違う
example
\end{exm}
\end{frame}

\section{文字の色}             %% 文字の色を変える
\begin{frame}
\frametitle{文字の色を変えてみよう}
{\color{red}赤}\pause
{\color{blue}青}\pause
{\color{green}緑}
\end{frame}

\end{document}
