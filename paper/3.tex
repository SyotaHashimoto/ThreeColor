% 三角形三色問題の証明の Coq の実装の準備
\section{ Coq に実装するための準備}
% 3.1 定義
% 関数 \mix・記号の説明
% WellColoredTriangle x y n c0 c1 c2 :=
% Cpos x y c0 /\ Cpos (x+n) y c1 /\ Cpos x (y+n) c2 -> c2 = \mix c0 c1
\subsection{定義}
三角形三色問題のような幾何的な直観に基づく問題や前節で述べたような証明をCoqに実装するには,図形の状況を表現する論理式を用意し,それらを用いて問題の暗黙の前提や色塗り規則などを公理化することで形式化する必要がある.
ここでは実装する際に用いた定義や公理について述べる.
\begin{dfn}[$\Color$]
  マスに塗る色の集合を次のように定義する.
  \[
  \Color \eqDef \{\red, \yel, \blu\}
  \]
  このとき,$\red$は{\rm{red}},$\yel$は{\rm{yellow}},$\blu$は{\rm{blue}}を表している.
  以降,$\red$を$r$,$\yel$は$y$,$\blu$は$b$として略記することもある.
\end{dfn}
\begin{dfn}[$\mix$]
  $\mix$ $:$ $(\Color \times \Color )$ $\to$ $\Color$ を以下で定義する.
  \[
  \begin{tabular}{ccccc}
    $(r,r) \Pto r,$ & $(r,y) \Pto b,$ & $(r,b) \Pto y,$ \\
    $(y,r) \Pto b,$ & $(y,y) \Pto y,$ & $(y,b) \Pto r,$ \\
    $(b,r) \Pto y,$ & $(b,y) \Pto r,$ & $(b,b) \Pto b.$ \\
  \end{tabular}
  \]
  演算 $\mix$ は塗り方の規則を再現するための関数である.
引数となる$2$色が同じ場合は同じ色を返し,異なる場合は$2$色とも異なる第三の色を返す関数である.
\end{dfn}
\begin{dfn}[$\colorYB$]
  $\colorYB$ $:$ $(\N \times \N \times \N)$ $\to$ $\Color$ を以下で定義する.

  $\colorYB (x,n,z) \eqDef$
  \[
  \begin{cases}
    \yel & (0 \leq z-x \leq n \land z-x\text{が奇数}) \\
    \blu & (0 \leq z-x \leq n \land z-x\text{が偶数}) \\
    \blu & (\text{otherwise})
  \end{cases}
  \]
  $\colorYB$は補題$\ref{lem:tri_nec}$の証明において$n$が偶数のときの最上段のマスの塗り方を表した関数である.
  一番左端にあるマスを基準(左から$x$番目のマス)としたときに,基準から右に$z$番目のマスを指定するときには$z-x$を用いて指定している.
  $z-x$は基準となるマスから離れているマス数を表しており,相対的に最上段のマスを指定している.
  また,$\colorYB$は最上段のマスを交互に塗っていることを$z-x$の偶奇によって再現している.
\end{dfn}
\begin{dfn}[$\colorYBBY$]
  $\colorYBBY$ $:$ $(\N \times \N \times \N)$ $\to$ $\Color$ を以下で定義する.

  $\colorYBBY(x,n,z) \eqDef$
  \[
  \begin{cases}
    \yel & (0 \leq z-x \leq n/2 \land z-x\text{が偶数}) \\
    \yel & (n/2+1 \leq z-x \leq n \land z-x\text{が奇数}) \\
    \blu & (0 \leq z-x \leq n/2 \land z-x\text{が奇数}) \\
    \blu & (n/2+1 \leq z-x \leq n \land z-x\text{が偶数}) \\
    \yel & (\text{otherwise})
  \end{cases}
  \]
  $\colorYBBY$は補題$\ref{lem:tri_nec}$の証明において$n$が奇数 かつ $3^{k} < n \leq 2 \cdot 3^{k}$のときの最上段のマスの塗り方を表した関数である.
  $\colorYB$と同様にして基準となるマスから離れているマス数を用いて相対的に最上段のマスを$1$つ指定している.
  また,$0 \leq z-x \leq n/2$の範囲では左端のマスから偶数番目のときは$\yel$,奇数番目のときは$\blu$を塗り,$n/2+1 \leq z-x \leq n$の範囲では塗る色が入れ替わる.
  このようにして,$\colorYBBY$は外側から内側に向かって対称的に最上段のマスを交互に塗っていることを$z-x$の偶奇によって再現している.
\end{dfn}
\begin{dfn}[$\colorBYB$]
  $\colorBYB$ $:$ $(\N \times \N \times \N \times \N )$ $\to$ $\Color$ を以下で定義する.

  $\colorBYB(x,n,k,z) \eqDef$
  \[
  \begin{cases}
    \yel & (3^k \leq z-x \leq n-3^k) \\
    \blu & (\text{otherwise})
  \end{cases}
  \]
  $\colorBYB$は補題$\ref{lem:tri_nec}$の証明において$n$が奇数かつ$2 \cdot 3^{k'} + 1 \leq n < 3^{k'+1}$のときの最上段のマスの塗り方を表した関数である.
  $\colorYBBY$は基準となるマスから右に$3^k$番目から$n-3^k$番目までマスの色を黄色で塗り,その他を青で塗ることで再現している.
\end{dfn}
\begin{dfn}[$\Cpos$]
  $x, y \in \N$,$c \in \Color$に対して,$\Cpos(x,y,c)$を以下の述語として定義する.
  
  $\Cpos(x,y,c)$ $\iffDef$
  左から$x$番目,上から$y$番目のマスに塗られている色が$c$である.
  
  さらに,逆三角形に配置されたマスにおいて左から$x$番目,上から$y$番目のマスの座標を$(x,y)$と表す.
  \footnote{
    マスの座標の表し方はCoqに実装しておらず$\Cpos$のみ実装している.
    }
\end{dfn}
\begin{exm}
  図$\ref{fig:nine_steps}$において,逆三角形の$3$つの端点のマスに関するそれぞれの命題$\Cpos(0,0,yel)$,$\Cpos(9,0,blu)$,$\Cpos(9,0,red)$は正しい.
\end{exm}
\begin{dfn}[$\WCT$]
  $x, y, n \in \N$,$c_0, c_1,$ $c_2 \in \Color$に対して,
  \WCT$(x, y, n, c_0,$ $c_1, c_2)$を次のように定義する.
  
  \WCT$(x, y, n, c0, c1, c2)$ $\iffDef$
  
  $(\Cpos(x,y,c_0)$ $\land$ $\Cpos(x+n,y,c_1)$ $\land$ $\Cpos(x,y+n,c_2))$ $\Imp$ $c_2 = \mix(c_0,c_1)$.
  
  $\WCT$は定義$\ref{dfn:wc_tri}$で述べた$n$段の調和彩色三角形の定義を$\Cpos$や$\mix$を用いて論理式に書き直したものである.
  $x$,$y$は逆三角形の左端のマス$(x,y)$を基準として定めるために用いており,$n$は逆三角形の一辺の長さを表しており,$3$つのマス$(x,y), (x+n,y), (x,y+n)$に塗られている色は調和性を満たしている$(c_2=\mix(c_0,c_1))$ことを表している.
\end{dfn}

% 3.2 公理
\subsection{公理}
ここからは三角形三色問題を再現するための$4$つ公理を述べる.
\begin{axm}[$\Cexists$] \label{axm:exists}
  $\forall$ x, y $\in$ $\N$に対して,$\exists$ $c$ $\in$ $\Color$, $\Cpos(x,y,c)$.
  
  この公理はすべてのマスには色が塗られていることを表している.
\end{axm}
\begin{axm}[$\Cuniq$] \label{axm:uniq}
  $\forall x, y \in \N, \forall c_0, c_1 \in \Color$に対して,
  $(\Cpos(x,y,c_0) \land \Cpos(x,y,c_1)) \Imp c_0 = c_1.$
  
  この公理は$1$つのマスに$2$色塗られているときは,その$2$色が同じ色であることを表している.
  すなわち,$1$つのマスに塗れる色は$1$色までであることを表している.
\end{axm}
\begin{axm}[$\Cmix$] \label{axm:mix}
  $\forall x, y \in \N$,$\forall c_0, c_1, c_2$ $\in$ $\Color$に対して,
  $(\Cpos(x,y,c_0)$ $\land$ $\Cpos(x+1,y,c_1)$ $\land$ $\Cpos(x,y+1,c_2))$ $\Imp$ $c_2 = \mix(c_0,c_1)$.
  
  この公理は隣接する$2$つのマスの色に演算$\mix$を適用すると間にある$1$段下のマスの色が決まるという三角形三色問題の規則をしている.
\end{axm}
\begin{axm}[$C\_paint$] \label{axm:paint}
  $\forall x, y, i \in \N, \forall f:\N \to Color$に対して,
  $Cpos(x+i,y,f(x+i))$.
  
  この公理における$f$は最上段のマスの塗り方を関数として表している.
  すなわち,最上段のマスはすべて好きな色を塗ることができることを表している.
\end{axm}

% 3.3 補題
% mixCut, AllRed, falseColor
\subsection{補題} \label{sec:lem}
次に証明を円滑に進めていくために用いた補題について述べる.
\begin{lem}[$\mixCut$] \label{lem:mixCut}
  $\forall c_0, c_1, c_2, c_3 \in Color$に対して,
  $\mix( \mix ( \mix(c_0,c_1) , \mix(c_1,c_2) ), \mix( \mix(c_1 c_2),$\\
  $\mix(c_2,c_3) ) )$ $=$ $\mix(c_0,c_3)$.

  $\mixCut$は演算$\mix$のもつ性質を論理式にしたものであり,$\mix$と$4$色を用いて表された色は$2$色のみを用いて書き換えることができることを表している.
  証明する際には各色が$3$通りずつ取り得るので合計$3^4=81$通りの場合分けをおこなって$\mix$の計算をすれば証明することができる.
  三角形三色問題の十分条件$(\text{補題}\ref{lem:tri_suf})$を証明する際に用いる補題である.

\end{lem}
\begin{lem}[$\AllRed$] \label{lem:AllRed}
  $\forall x, y, n\in \N$ に対して,
  $(\forall i. \in \N,$ $(0 \leq i \leq n$ $\Imp$ $\Cpos(x+i,y,red)))$ $\Imp$ $\Cpos(x,y+n,red)$. 

  三角形三色問題の必要条件$(\text{補題}\ref{lem:tri_nec})$の証明において,$n$がどの場合でもすべてのマスが赤に塗られている段があることに帰着させて矛盾を導いている.
  AllRedにより,すべてのマスが赤で塗られている段があるときは最下段のマスは赤であることを推測できるという補題である.
\end{lem}
\begin{lem}[$\falseColor$] \label{lem:falseColor}
  $\forall x, y \in \N,$ $\forall c_0, c_1$ $\in$ $\Color$に対して,
  $(c_0 \neq c_1 \land  \Cpos(x,y,c_0) \land \Cpos(x,y,c_1))$ $\Imp$ $\bot$.

  falseColorは$3^2=9$通りの場合分けと公理$\ref{axm:uniq}$より証明できる.
  同じマスに対して異なる色が塗れてしまっているときには矛盾するという補題である.
  三角形三色問題の必要条件$(\text{補題}\ref{lem:tri_nec})$の証明において,最下段の色が赤で塗られることと補題$\ref{lem:tri_suf}$より矛盾を導くときに用いる.
\end{lem}

