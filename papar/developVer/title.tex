% 論文のタイトル
\title{
  Coq による三角形三色問題の証明
}

\author{橋本 翔太 木村 大輔
  %
  % ここにタイトル英訳 (英文の場合は和訳) を書く.
  \ejtitle{
    A formal proof for the three-colored triangle problem on Coq
  }
  %
  % 所属 (和文および英文) を書く.複数著者の所属はまとめてよい.
  \shozoku{Shota Hashimoto}{東邦大学大学院理学研究科}%
  {Toho University}  
  \shozoku{Daisuke Kimura}{東邦大学大学院理学研究科}%
  {Toho University}
  %
  \shutten
  \uketsuke{2099}{99}{99} % dummy
}


\Jabstract{
  Coq とは数学の証明作成を支援する定理証明支援系である.人間は Coq と対話的に証明作成を行うことで誤りを排除した信頼できる証明を得ることができる. 三角形三色問題とは,$n$ 段の逆三角形に配置された六角形のマスに対して,互いに隣接する任意の3マスの色が全て同じかどれも異なるように逆三角形を3色で塗り分けたとき,逆三角形の3頂点のマスの色が必ず全て同じ,もしくはどれも異なるような段数の一般項を求める問題である.この問題は雑誌「数学セミナー」の「エレガントな解答もとむ」欄に出題されており,一般項は $3^k$ 段の形で表せることが示されている.本研究では,数学セミナーでの証明をCoqで形式化して証明を完成させた. これにより,幾何的な直観に頼った側面のあった元の議論を論理に基づいた形式的な証明に直すことができた.
}

\maketitle \thispagestyle {empty}
