% 三角形三色問題の概要(数学)
\section{Coqにおける三角形三色問題}

% 4.1 三角形三色問題の証明の Coq の実装(=>)
\subsection{十分条件}
補題\ref{lem:tri_suf}をCoqに実装するために論理式の形にしたものが次の定理\ref{thm:tri_suf}である.
\begin{thm} \label{thm:tri_suf}
  $\forall k, n, x, y \in \N, c_0 c_1 c_2 \in Color, n = 3 ^ k \Imp WellColoredTriangle(x,y,n,c_0,c_1,c_2)$.
\end{thm}
\begin{proof}
  $k$に関する数学的帰納法を用いて証明する.
  $k=0$のときは$n=1$となるので明らかに成立する.
  次に$k$のとき成立すると仮定して$k+1$のときも成立することを示す.
  まず最初に,\ref{fig:suf_steps}における10色の$c^x_y$が存在することを
  公理\ref{axm:exists}より示す.
  例えば,基準となるマス$(x,y)$から$3^k$マス右にあるマス$(x+3^k,y)$に色が存在していることを示すときには,\verb|exists| \verb|c| \verb|:| \verb|Color,| \verb|Cpos| \verb|(x+3^k)| \verb|y| \verb|c.| として新たにサブゴールをつくる.
  これは \verb|by| \verb|apply| \verb|C_exists.| をすればよい.
  以降,証明して得られた命題にあるそれぞれのマスに存在する色を図\ref{fig:suf_steps}のように名前をつけておく.
  \begin{figure}[h]
    \centering
    % 3^{k'+1} 段の三角形
\begin{tikzpicture}
  {\normalsize{
      % 3^(k+1)段目
      \node (a0) {$c_{2}$};
      % 2*3^k段目
      \node[above left=0.5cm of a0] (b0) {$c_{8}$};
      \node[above right=0.5cm of a0] (b1) {$c_{9}$};
      % 3^k段目
      \node[above left=0.5cm of b0] (c0) {$c_{5}$};
      \node[above right=0.5cm of b0] (c1) {$c_{6}$};
      \node[above right=0.5cm of b1] (c2) {$c_{7}$};
      % 0段目
      \node[above left=0.5cm of c0] (d0) {$c_{0}$};
      \node[above right=0.5cm of c0] (d1) {$c_{3}$};
      \node[above left=0.5cm of c2] (d2) {$c_{4}$};
      \node[above right=0.5cm of c2] (d3) {$c_{1}$};
      % 点線 [3^(k+1)段 〜 2*3^k 段]
      \node[blue] at ($(a0)!.5!(b0)$) {$\ddots$};
      \node[blue] at ($(b0)!.5!(b1)$) {$\cdots$};
      \node[blue] at ($(a0)!.5!(b1)$) {$\iddots$};
      % 点線 [2*3^k 段 〜 3^k 段]
      \node[teal] at ($(b0)!.5!(c0)$) {$\ddots$};
      \node[teal] at ($(c0)!.5!(c1)$) {$\cdots$};
      \node[teal] at ($(b0)!.5!(c1)$) {$\iddots$};
      \node[red] at ($(b1)!.5!(c1)$) {$\ddots$};
      \node[red] at ($(c1)!.5!(c2)$) {$\cdots$};
      \node[red] at ($(b1)!.5!(c2)$) {$\iddots$};
      % 点線 [3^k 段 〜 0 段]
      \node[red] at ($(c0)!.5!(d0)$) {$\ddots$};
      \node[red] at ($(d0)!.5!(d1)$) {$\cdots$};
      \node[red] at ($(c0)!.5!(d1)$) {$\iddots$};
      \node[blue] at ($(c1)!.5!(d1)$) {$\ddots$};
      \node[blue] at ($(d1)!.5!(d2)$) {$\cdots$};
      \node[blue] at ($(c1)!.5!(d2)$) {$\iddots$};
      \node[teal] at ($(c2)!.5!(d2)$) {$\ddots$};
      \node[teal] at ($(d2)!.5!(d3)$) {$\cdots$};
      \node[teal] at ($(c2)!.5!(d3)$) {$\iddots$};
  }}
\end{tikzpicture}

    \caption{三色三角形($n=3^{k'+1}$のとき)}
    \label{fig:suf_steps}
  \end{figure} 
  次に$6$個の$3^k$段の調和三角形三角形があることを示す.
  例えば,図\ref{fig:suf_steps}における$3$つの端点$\left(c_{0},c_{3},c_{5}\right)$をもつ$3^k$段の逆三角形は,\verb|WellColoredTriangle| \verb|x| \verb|y| \verb|(3.^k)| \verb|c0| \verb|c3| \verb|c5.| として新たにサブゴールをつくる.
  これは$k$に関する数学的帰納法の仮定(\verb|IHk|)からすぐに示せるので \verb|by| \verb|apply| \verb|IHk.| をすればよい.
  すると,ここまでに証明した$10$個の$Cpos$に関する命題と$6$個の$3^k$段の調和三角形三角形に関する命題から命題\verb|mix| \verb|c0| \verb|c1| \verb|=| \verb|mix| \verb|(mix| \verb|(mix| \verb|c0| \verb|c3)| \verb|(mix| \verb|c3| \verb|c4))| \verb|(mix| \verb|(mix| \verb|c3| \verb|c4)| \verb|(mix| \verb|c4| \verb|c1))| にサブゴールを書き換える.
  最後に$mixCut$を適用させることで証明を終了させる.
\end{proof}

% 4.2 三角形三色問題の証明の Coq の実装(<=)
% n が偶数の場合,n が奇数で短い場合,n が奇数で長い場合
% $n$が偶数のとき
% $n$が奇数 かつ $3^{k'} < n \leq 2 \cdot 3^{k'}$のとき
% $n$が奇数 かつ $2 \cdot 3^{k'} + 1 \leq n < 3^{k'+1}$のとき

\subsection{必要条件}
